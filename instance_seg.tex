\newpage
\clearpage
\section{Segmentação de Instancias}
\label{instance:instance}
%%%%%%%%%%%%%%%%%%%%%%%%%%%%%%%%%%%%%%%%%%%%%%%%%%%%%%%%%%%%%%%%%%%%%%%%%%%%%%%%%%%%%%%%%%%%%%%%%%%%%
\begin{itemize}
    \item Falar sobre o inicio com \cite{Vaillant1994}
    \item citar \cite{Minaee2021}
    \item citar \cite{Bolya2019}
    \item falar sobre estado da arte \cite{Hafiz2020}
\end{itemize}
%%%%%%%%%%%%%%%%%%%%%%%%%%%%%%%%%%%%%%%%%%%%%%%%%%%%%%%%%%%%%%%%%%%%%%%%%%%%%%%%%%%%%%%%%%%%%%%%%%%%%
Sendo assim, nessa seção estaremos tratando das métricas utilizadas para calcular o desempenho de modelos que utilizam de segmentação de instâncias (seção \ref{instance:metrics}), bem como o modelo que representa o estado da arte para esse tipo de atividade (seção \ref{instance:mask}).


\subsection{Métricas}
\label{instance:metrics}


\subsubsection{\textit{Average Precision}(AP)}
\label{instance:AP}
%%%%%%%%%%%%%%%%%%%%%%%%%%%%%%%%%%%%%%%%%%%%%%%%%%%%%%%%%%%%%%%%%%%%%%%%%%%%%%%%%%%%%%%%%%%%%%%%%%%%%
\begin{itemize}
    \item citar \cite{Hariharan2014}
    \item citar \cite{Lin2014}
\end{itemize}
%%%%%%%%%%%%%%%%%%%%%%%%%%%%%%%%%%%%%%%%%%%%%%%%%%%%%%%%%%%%%%%%%%%%%%%%%%%%%%%%%%%%%%%%%%%%%%%%%%%%%


\subsubsection{Mask R-CNN}
\label{instance:mask}
that can create a pixel-wise mask for each object in an image.
Mask R-CNN is an extension of the popular Faster R-CNN object detection architecture.
The Faster R-CNN method generates two things for each object in the image:
 - Its class
 - The bounding box coordinates
Mask R-CNN adds a third branch to this which outputs the object mask as well.
It also returns the mask for each proposal.
Mask R-CNN is the current state-of-the-art for image segmentation and runs at 5 fps.

%%%%%%%%%%%%%%%%%%%%%%%%%%%%%%%%%%%%%%%%%%%%%%%%%%%%%%%%%%%%%%%%%%%%%%%%%%%%%%%%%%%%%%%%%%%%%%%%%%%%%
\begin{itemize}
    \item Ler site (https://www.analyticsvidhya.com/blog/2019/07/computer-vision-implementing-mask-r-cnn-image-segmentation/) e tirar mais informações;
    \item citar \cite{He2020};
    \item colocar imagem do modelo (https://cdn.analyticsvidhya.com/wp-content/uploads/2019/03/Mask-R-CNN.png) no final e conferir autoria (\cite{He2020}) para citar.
\end{itemize}
%%%%%%%%%%%%%%%%%%%%%%%%%%%%%%%%%%%%%%%%%%%%%%%%%%%%%%%%%%%%%%%%%%%%%%%%%%%%%%%%%%%%%%%%%%%%%%%%%%%%%

\subsection{Considerações Finais da Seção}
\label{instance:conclusion}

Por meio dos estudos realizados em relação a segmentação de instancias, pôde-se perceber que há uma recomendação para fatores como ..., não obstante,como dito por \cite{Kirillov2019a}, as segmentações de instancias possuem uma deficiencia 