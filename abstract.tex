\section*{Abstract}
In the field of computer vision, activities related to image segmentation have provided advances in more accurate medical analysis, scene understanding, autonomous systems design, among other similar studies, which have gained amplitude due to the advent of artificial neural networks and deep learning techniques, which provide a basis for the development of many models and architectures that aim to reach the state-of-the-art, providing better performance for image segmentation issues.
Among the applications in the segmentation area, the dental field stands out, in which studies and tools - from traditional to the most modern ones - have been developed with the goal of helping professionals in the area, providing agility, detail and assertiveness regarding the dental situation of each patient, such as the number of teeth present, teeth absent, caries presence, area of tooth compromise, and visual components present in the mouth. 
Among the vast amounts of segmentation works produced with application in the dental field,  it is notoriously low the amount of works that explore the most modern segmentations, which are able to provide an approximation in what concerns human understanding and actually help in details that may go unnoticed, thus, in fact, providing a greater wealth of information and semantics to the scenes analyzed.
Therefore, the present work proposes to present an overview about artificial neural networks concepts and to contextualize traditional and modern segmentation techniques, presenting their advantages and disadvantages through some problems, aiming to address: \rom{1}) the application of a modern segmentation in dental visual components describing a scene hierarchically and \rom{2}) changing the pooling layer of the base models of modern segmentation, in order to contribute with dental professionals in their analysis and increase the evaluation metrics of the models, respectively.
Finally, it is hoped that this proposal can contribute to the development of an automatic oral health analysis tool, which can be used by professionals or individuals to get an idea of the patient's oral condition, thus allowing access to an initial dental analysis and monitoring of the population's oral health.